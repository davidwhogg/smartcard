\documentclass[letterpaper,12pt]{article}
\setlength{\headheight}{0ex}
\setlength{\headsep}{0ex}
\input{hogg_nsf}
\pagestyle{empty}
\begin{document}

\paragraph{David W. Hogg} http://cosmo.nyu.edu/hogg/

\paragraph{Professional Preparation:}~\\
\begin{tabular}{llll}
\textsl{undergraduate}
  & Massachusetts Institute of Technology & Physics & SB 1992\\
\textsl{graduate}
  & California Institute of Technology    & Physics & PhD 1998\\
\textsl{postdoctoral}
  & Institute for Advanced Study          & Astrophysics & 1997--2001
\end{tabular}
           
\paragraph{Appointments:}~\\
\begin{tabular}{lll}
Associate Professor & Department of Physics & New York University, 2007--present\\
Assistant Professor & Department of Physics & New York University, 2001--2007\\
Long-term member & Astrophysics & Institute for Advanced Study, 1997--2001 \\
\end{tabular}

\paragraph{Honors:}
$\bullet$~New York University ``Golden Dozen'' Teaching Award (2004);
$\bullet$~Princeton University Engineering Council Teaching Award (2000);
$\bullet$~Caltech Undergraduate Teaching Award, Associated Students of Caltech (1996)
$\bullet$~Phi Beta Kappa (1992);
$\bullet$~Sigma Xi, (1992);
$\bullet$~Award of merit, International Physics Olympiad (1988).

\paragraph{Recent Service:}
$\bullet$~\Spitzer\ Science Center Oversight Committee, (2008--present); 
$\bullet$~\project{NASA Extragalactic Database} Users Committee (2006--present);
$\bullet$~Panel Chair, \Spitzer\ Time Allocation Committee (2005);
$\bullet$~Leader, \project{Sloan Digital Sky Survey} Calibration Task Force (2000--2003);
$\bullet$~National Optical Astronomy Observatories Time Allocation Committee (2000--2002);

\paragraph{Graduate Students Advised:}
$\bullet$~Jo Bovy (PhD 2011);
$\bullet$~Daniel Foreman-Mackey (current student);
$\bullet$~Fengji Hou (current student; co-advised by Jonathan Goodman, NYU);
$\bullet$~Tao Jiang (current student);
$\bullet$~Dustin Lang (PhD 2009; co-advised by Sam Roweis, Toronto);
$\bullet$~Morad Masjedi (PhD 2007);
$\bullet$~Ronin RongYing Wu (PhD 2011);
$\bullet$~Adi Zolotov (PhD 2011; co-advised by Beth Willman, Haverford).

\paragraph{Publications Related to the Proposed Project:}
\begin{itemize}\setlength{\itemsep}{0pt}
\item
Hou,~F., Goodman,~J., Weare,~J., \& Schwab,~C., 2011, An
Affine-Invariant Sampler for Exoplanet Fitting and Discovery in Radial
Velocity Data, \textit{Astrophys.\,J.}\ submitted (arXiv:1104.2612).
\item
Bovy,~J., Hogg,~D.~W., \& Roweis,~S., 2011,
Extreme deconvolution:\ Inferring complete distribution functions
from noisy, heterogeneous and incomplete observations,
\textit{Ann.\,Appl.\,Stat.}\ in press (arXiv:0905.2979).
\item
Bovy,~J., Hennawi,~J.~F., Hogg,~D.~W., \etal, 2011,
Think outside the color-box:\ Probabilistic target selection and
the \project{SDSS-XDQSO} quasar targeting catalog,
\textit{Astrophys.\,J.}\ \textbf{729} 141.
\item
Hogg,~D.~W., Myers,~A.~D., \& Bovy,~J., 2010, Inferring the
eccentricity distribution, \textit{Astrophys.\,J.}\ \textbf{725}
2166--2175.
\item
Lang,~D., Hogg,~D.~W., Mierle,~K., Blanton,~M., \& Roweis,~S., 2010,
\textsl{Astrometry.net}: Blind astrometric calibration of arbitrary
astronomical images,
\textit{Astron.\,J.}\ \textbf{139} 1782--1800.
\item
Lang,~D., Hogg,~D.~W., Jester,~S., \& Rix,~H.-W., 2009, Measuring the
undetectable:\ Proper motions and parallaxes of very faint sources,
\textit{Astron.\,J.}\ \textbf{137} 4400--4411.
\item
Hogg,~D.~W., Finkbeiner,~D.~P., Schlegel,~D.~J. \& Gunn,~J.~E., 2001, A
photometricity and extinction monitor at the Apache Point Observatory,
\textit{Astron.\,J.}\ \textbf{122} 2129--2138.
\end{itemize}

\paragraph{Other Significant Publications:}
\begin{itemize}\setlength{\itemsep}{0pt}
\item
Masjedi,~M., Hogg,~D.~W., \etal, 2006, Very small-scale clustering and
merger rate of luminous red galaxies, \textit{Astrophys.\,J.}\ 
\textbf{644} 54--60.
\item
Eisenstein,~D.~J., Zehavi,~I., Hogg,~D.~W., \etal, 2005,
Detection of the baryon acoustic peak in the large-scale correlation
function of Sloan Digital Sky Survey Luminous Red Galaxies,
\textit{Astrophys.\,J.}\ \textbf{633} 560--574.
\item
Willman,~B. \etal, 2005, A new Milky Way dwarf galaxy in Ursa Major,
\textit{Astrophys.\,J.\,Lett.}\ \textbf{626} L85--L88.
\item
Hogg,~D.~W. \etal, 2004, The dependence on environment of the
color--magnitude relation of galaxies, \textit{Astrophys.\,J.\,Lett.}\ 
\textbf{601} L29--L32.
\item
Cohen,~J.~G., Hogg,~D.~W., \etal, 2000, Caltech Faint
Galaxy Redshift Survey.\ X.\ A redshift survey in the region of the
Hubble Deep Field North, \textit{Astrophys.\,J.}\ \textbf{538} 29--52.
\end{itemize}

\paragraph{Management of prior NASA projects}
\begin{itemize}\setlength{\itemsep}{0pt}
\item 
PI of NASA Astrophysics Data Analysis Grant NNX08AJ48G,
\textit{Multi-wavelength astrometric catalog built from NASA data,}
2008--2011.
\item
PI of NASA Long-Term Space Astrophysics Grant NAG5-11669,
\textit{Tools for Galaxy Astrophysics in the Era of the
\project{Space Infrared Telescope Facility,}}
2002--2007.
\item
PI of NASA \project{Spitzer Space Telescope} General Observer Grant
20120, \textit{A search
for PAH emission in extremely low luminosity galaxies,}
2005--2007.
\end{itemize}

\paragraph{Relevant expertise and accomplishments:}
\begin{itemize}\setlength{\itemsep}{0pt}
\item
Co-wrote and currently run \project{Astrometry.net} to calibrate
astronomical images; currently have more than one thousand users and
an open-source code base.
\item
Co-author of open-source code for estimating densities
given heterogeneous observations
(http://code.google.com/p/extreme-deconvolution/), for performing
classification and target selection
http://code.google.com/p/xdtarget/, and for performing very fast Monte
Carlo sampling in inference contexts (arXiv:1104.2612).
\item
Introducer of hierarchical Bayesian methods into astronomy, in an
exoplanet context (\textit{Astrophys.\,J.}\ \textbf{725} 2166) and in
a quasar target-selection context
(\textit{Astrophys.\,J.}\ \textbf{729} 141).  The latter is the basis
for the \project{SDSS-III BOSS} ``core'' quasar target selection.
\item
Leader of Photometric Calibration Task Force to investigate and help
solve photometric calibration issues for the Sloan Digital Sky Survey.
Executed a full re-calibration of the Survey based on principles
introduced from machine learning.  Co-I on large projects to
re-distribute scientifically vetted subsamples of Survey data (the
\project{NYU-VAGC}).
\item
Author of pedagogical texts on arXiv.org for cosmology, fundamental
measurement in astronomy, and---recently---statistical inference.
These texts are associated with publicly released code.
\end{itemize}

\end{document}
