% this file is part of the smartcard project

\documentclass[letterpaper,12pt]{article}
\setlength{\headheight}{0ex}
\setlength{\headsep}{0ex}
%----- exact 1-in margins
% NB: headheight and headsep MUST exist and be set
\setlength{\textwidth}{6.5in}
\setlength{\textheight}{9in}
\addtolength{\textheight}{-1.0\headheight}
\addtolength{\textheight}{-1.0\headsep}
\setlength{\topmargin}{0.0in}
\setlength{\oddsidemargin}{0.0in}
\setlength{\evensidemargin}{0.0in}

%----- typeset certain kinds of words
\newcommand{\observatory}[1]{\textsl{#1}}
\newcommand{\package}[1]{\textsf{#1}}
\newcommand{\project}[1]{\package{#1}}
\newcommand{\an}{\package{Astrometry.net}}
\newcommand{\NASA}{\observatory{NASA}}
\newcommand{\Kepler}{\observatory{Kepler}}
\newcommand{\kepler}{\Kepler}
\newcommand{\MAST}{\observatory{MAST}}
\newcommand{\EA}{\observatory{Exoplanet Archive}}
\newcommand{\TESS}{\observatory{TESS}}
\newcommand{\galex}{\observatory{GALEX}}
\newcommand{\Spitzer}{\observatory{Spitzer}}
\newcommand{\gaia}{\observatory{Gaia}}
\newcommand{\lsst}{\observatory{LSST}}
\newcommand{\sdss}{\observatory{SDSS}}
\newcommand{\latin}[1]{\textit{#1}}
\newcommand{\eg}{\latin{e.g.}}
\newcommand{\etal}{\latin{et~al.}}
\newcommand{\etc}{\latin{etc.}}
\newcommand{\ie}{\latin{i.e.}}
\newcommand{\vs}{\latin{vs.}}

%----- typeset journals
% \newcommand{\aj}{Astron.\,J.}
% \newcommand{\apj}{Astrophys.\,J.}
% \newcommand{\apjl}{Astrophys.\,J.\,Lett.}
% \newcommand{\apjs}{Astrophys.\,J.\,Supp.\,Ser.}
% \newcommand{\mnras}{Mon.\,Not.\,Roy.\,Ast.\,Soc.}
% \newcommand{\aap}{Astron.\,\&~Astrophys.}

%----- Tighten up paragraphs and lists
\setlength{\parskip}{0.0ex}
\setlength{\parindent}{0.2in}
\renewenvironment{itemize}{\begin{list}{$\bullet$}{%
  \setlength{\topsep}{0.0ex}%
  \setlength{\parsep}{0.0ex}%
  \setlength{\partopsep}{0.0ex}%
  \setlength{\itemsep}{0.0ex}%
  \setlength{\leftmargin}{1.5\parindent}}}{\end{list}}
\newcounter{actr}
\renewenvironment{enumerate}{\begin{list}{\arabic{actr}.}{%
  \usecounter{actr}%
  \setlength{\topsep}{0.0ex}%
  \setlength{\parsep}{0.0ex}%
  \setlength{\partopsep}{0.0ex}%
  \setlength{\itemsep}{0.0ex}%
  \setlength{\leftmargin}{1.5\parindent}}}{\end{list}}

%----- mess with paragraph spacing!
\makeatletter
\renewcommand\paragraph{\@startsection{paragraph}{4}{\z@}%
                                    {1ex}%
                                    {-1em}%
                                    {\normalfont\normalsize\bfseries}}
\makeatother

%----- Special Hogg list for references
  \newcommand{\hogglist}{%
    \rightmargin=0in
    \leftmargin=0.25in
    \topsep=0ex
    \partopsep=0pt
    \itemsep=0ex
    \parsep=0pt
    \itemindent=-1.0\leftmargin
    \listparindent=\leftmargin
    \settowidth{\labelsep}{~}
    \usecounter{enumi}
  }

%----- side-to-side figure macro
%------- make numbers add up to 94%
 \newlength{\figurewidth}
 \newlength{\captionwidth}
 \newcommand{\ssfigure}[3]{%
   \setlength{\figurewidth}{#2\textwidth}
   \setlength{\captionwidth}{\textwidth}
   \addtolength{\captionwidth}{-\figurewidth}
   \addtolength{\captionwidth}{-0.02\figurewidth}
   \begin{figure}[htb]%
   \begin{tabular}{cc}%
     \begin{minipage}[c]{\figurewidth}%
       \resizebox{\figurewidth}{!}{\includegraphics{#1}}%
     \end{minipage} &%
     \begin{minipage}[c]{\captionwidth}%
       \textsf{\caption[]{\footnotesize {#3}}}%
     \end{minipage}%
   \end{tabular}%
   \end{figure}}

%----- top-bottom figure macro
 \newlength{\figureheight}
 \setlength{\figureheight}{0.75\textheight}
 \newcommand{\tbfigure}[2]{%
   \begin{figure}[htp]%
   \resizebox{\textwidth}{!}{\includegraphics{#1}}%
   \textsf{\caption[]{\footnotesize {#2}}}%
   \end{figure}}

%----- deal with pdf page-size stupidity
\special{papersize=8.5in,11in}
\setlength{\pdfpageheight}{\paperheight}
\setlength{\pdfpagewidth}{\paperwidth}

% no more bad lines!
\sloppy

\pagestyle{empty}
\begin{document}

\paragraph{David W. Hogg} http://cosmo.nyu.edu/hogg/

\paragraph{Professional Preparation:}~\\[3pt]
\begin{tabular}{llll}
\textsl{undergraduate}
  & Massachusetts Institute of Technology & Physics & SB 1992\\
\textsl{graduate}
  & California Institute of Technology    & Physics & PhD 1998\\
\textsl{postdoctoral}
  & Institute for Advanced Study          & Astrophysics & 1997--2001
\end{tabular}
           
\paragraph{Appointments:}~\\[3pt]
\begin{tabular}{lll}
Associate Professor & Department of Physics & New York University, 2007--present\\
Assistant Professor & Department of Physics & New York University, 2001--2007\\
Long-term member & Astrophysics & Institute for Advanced Study, 1997--2001 \\
\end{tabular}

\paragraph{Honors:}
$\bullet$~NYU ``Golden Dozen'' Teaching Award (2004);
$\bullet$~Princeton University Engineering Council Teaching Award (2000);
$\bullet$~Caltech Undergraduate Teaching Award (1996);
$\bullet$~Phi Beta Kappa (1992);
$\bullet$~Sigma Xi, (1992)

\paragraph{Recent Service:}
$\bullet$~Executive Director, \project{Moore--Sloan Data Science Environment at NYU} (2014--present);
$\bullet$~\Spitzer\ Science Center Oversight Committee, (2008--present); 
$\bullet$~\project{NASA Extragalactic Database} Users Committee (2006--2013);
$\bullet$~Panel Chair, \Spitzer\ Time Allocation Committee (2005);
$\bullet$~Leader, \project{Sloan Digital Sky Survey} Calibration Task Force (2000--2003);

\paragraph{Graduate Students Advised:}
$\bullet$~Jo Bovy (PhD 2011);
$\bullet$~Daniel Foreman-Mackey (current student);
$\bullet$~Fengji Hou (PhD 2014; co-advised by Jonathan Goodman, NYU)
$\bullet$~Tao Jiang (PhD 2012);
$\bullet$~Dustin Lang (PhD 2009; co-advised by Sam Roweis, Toronto);
$\bullet$~Morad Masjedi (PhD 2007);
$\bullet$~Ronin RongYing Wu (PhD 2011);
$\bullet$~Mohammadjavad Vakili (current student);
$\bullet$~Adi Zolotov (PhD 2011; co-advised by Beth Willman, Haverford).

\paragraph{Publications Related to the Proposed Project:}
\begin{itemize}\setlength{\itemsep}{0pt}
\item
Ambikasaran,~S., Foreman-Mackey,~D., Greengard,~L., Hogg,~D.~W., \& O'Neil,~M.,
{Fast direct methods for Gaussian Processes and the analysis of NASA \Kepler\ Mission Data},
\textit{IEEE Transactions on Pattern Analysis and Machine Intelligence}, submitted (\textit{arXiv}:{1403.6015}).
\item
Hogg,~D.~W. \etal, 2013,
{Maximizing \Kepler\ science return per telemetered pixel:\ Detailed models of the focal plane in the two-wheel era},
\textit{arXiv}:{1309.0653}.
\item
Foreman-Mackey,~D., Hogg,~D.~W., Lang,~D., \& Goodman,~J., 2013,
{\project{emcee}:\ The MCMC Hammer},
\textit{Pubs.\,Astr.\,Soc.\,Pac.}\ \textbf{125} 306--312.
\item
Holmes,~R., Hogg,~D.~W., \& Rix,~H.-W., 2012,
{Designing imaging surveys for a retrospective relative photometric calibration},
\textit{Pubs.\,Astr.\,Soc.\,Pac.}\ \textbf{124} 1219--1231.
\item
Hou,~F., Goodman,~J., Weare,~J., \& Schwab,~C., 2011, An
Affine-Invariant Sampler for Exoplanet Fitting and Discovery in Radial
Velocity Data, \textit{Astrophys.\,J.}\ submitted (arXiv:1104.2612).
\item
Hogg,~D.~W., Myers,~A.~D., \& Bovy,~J., 2010, Inferring the
eccentricity distribution, \textit{Astrophys.\,J.}\ \textbf{725}
2166--2175.
\item
Lang,~D., Hogg,~D.~W., Jester,~S., \& Rix,~H.-W., 2009, Measuring the
undetectable:\ Proper motions and parallaxes of very faint sources,
\textit{Astron.\,J.}\ \textbf{137} 4400--4411.
\item
Padmanabhan,~N., \etal, 2008,
{An improved photometric calibration of the \project{Sloan Digital Sky Survey} imaging data},
\textit{Astrophys.\,J.}\ \textbf{674} 1217--1233.
\end{itemize}

\paragraph{Other Significant Publications:}
\begin{itemize}\setlength{\itemsep}{0pt}
\item
Bovy,~J., Hennawi,~J.~F., Hogg,~D.~W., \etal, 2011,
Think outside the color-box:\ Probabilistic target selection and
the \project{SDSS-XDQSO} quasar targeting catalog,
\textit{Astrophys.\,J.}\ \textbf{729} 141.
\item
Lang,~D., Hogg,~D.~W., Mierle,~K., Blanton,~M., \& Roweis,~S., 2010,
{\project{Astrometry.net}:\ Blind astrometric calibration of arbitrary astronomical images},
\textit{Astron.\,J.}\ \textbf{139} 1782--1800.
\item\label{pub:Masjedi2006}
Masjedi,~M., Hogg,~D.~W., \etal, 2006,
{Very small-scale clustering and merger rate of luminous red galaxies},
\textit{Astrophys.\,J.}\ \textbf{644} 54--60.
\item
Eisenstein,~D.~J., Zehavi,~I., Hogg,~D.~W., \etal, 2005,
Detection of the baryon acoustic peak in the large-scale correlation
function of Sloan Digital Sky Survey Luminous Red Galaxies,
\textit{Astrophys.\,J.}\ \textbf{633} 560--574.
\item
Willman,~B. \etal, 2005, A new Milky Way dwarf galaxy in Ursa Major,
\textit{Astrophys.\,J.\,Lett.}\ \textbf{626} L85--L88.
\item
Hogg,~D.~W. \etal, 2004, The dependence on environment of the
color--magnitude relation of galaxies, \textit{Astrophys.\,J.\,Lett.}\ 
\textbf{601} L29--L32.
\item
Cohen,~J.~G., Hogg,~D.~W., \etal, 2000, Caltech Faint
Galaxy Redshift Survey.\ X.\ A redshift survey in the region of the
Hubble Deep Field North, \textit{Astrophys.\,J.}\ \textbf{538} 29--52.
\end{itemize}

\paragraph{Management of prior NASA projects}
\begin{itemize}\setlength{\itemsep}{0pt}
\item
PI of NASA Astrophysics Data Analysis Grant NNX12AI50G,
\textit{The Lives and Deaths of Planets and Stars in the Value-Added UV Photon Catalog},
2012--2015.
\item 
PI of NASA Astrophysics Data Analysis Grant NNX08AJ48G,
\textit{Multi-wavelength astrometric catalog built from NASA data},
2008--2011.
\item
PI of NASA Long-Term Space Astrophysics Grant NAG5-11669,
\textit{Tools for Galaxy Astrophysics in the Era of the
\project{Space Infrared Telescope Facility}},
2002--2007.
\item
PI of NASA \project{Spitzer Space Telescope} General Observer Grant
20120, \textit{A search
for PAH emission in extremely low luminosity galaxies},
2005--2007.
\end{itemize}

\paragraph{Relevant expertise and accomplishments:}
\begin{itemize}\setlength{\itemsep}{0pt}
\item
Co-wrote and currently run \project{Astrometry.net} to calibrate
astronomical images; currently have more than one thousand users and
an open-source code base.
\item
Co-author of open-source code for estimating densities
given heterogeneous observations
(http://code.google.com/p/extreme-deconvolution/), for performing
classification and target selection
http://code.google.com/p/xdtarget/, and for performing self-tuned Monte
Carlo sampling in inference contexts http://dan.iel.fm/emcee/.
\item
Introducer of hierarchical Bayesian methods into astronomy, in an
exoplanet context (\textit{Astrophys.\,J.}\ \textbf{725} 2166) and in
a quasar target-selection context
(\textit{Astrophys.\,J.}\ \textbf{729} 141).  The latter is the basis
for the \project{SDSS-III BOSS} ``core'' quasar target selection.
\item
Leader of Photometric Calibration Task Force to investigate and help
solve photometric calibration issues for the Sloan Digital Sky Survey.
Executed a full re-calibration of the Survey based on principles
introduced from machine learning.  Co-I on large projects to
re-distribute scientifically vetted subsamples of Survey data (the
\project{NYU-VAGC}).
\item
Author of pedagogical texts on arXiv.org for cosmology, fundamental
measurement in astronomy, and---recently---statistical inference.
These texts are associated with publicly released code.
\end{itemize}

\end{document}
