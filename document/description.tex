% this file is part of the Smart Card project.

% to-do
% -----
% - outline
% - write
% - submit
% - spend

\documentclass[letterpaper,12pt]{article}
\usepackage{epsfig}
\usepackage{graphicx}
\newcommand{\sectionname}{Section}
\setlength{\headheight}{2ex}
\setlength{\headsep}{3ex}
\input{hogg_nasa}
\newcommand{\Untrendy}{\project{Untrendy}}
\newcommand{\turnstyle}{\project{turnstyle}}
\newcommand{\bart}{\project{Bart}}
\newcommand{\TheCreator}{\project{The~Creator}}
\pagestyle{myheadings}
\markright{\textsf{\small Hogg \& Foreman-Mackey / probabilistic modeling of Kepler data}}
\begin{document}

\section*{project summary}

The \Kepler\ observatory has provided the most important and
scientifically productive data set ever for the study of exoplanets;
it has yeilded enormous numbers of new exoplanets and exoplanetary
systems.  It has generated unprecedented discoveries, measurements,
and insights in the domain of exoplanet science.  This proposal is to
systematically re-analyze all of the \Kepler\ data from the ground up,
to verify and improve upon the tremendously successful \Kepler-team
led analyses of the data, and to provide new inference capabilities to
the entire astrophysics community.  In particular, the proposal is to
design, build, and release to the public novel, easy-to-use tools for
probabilistic inference of exoplanet populations in the \Kepler data.

This project with design, build, and release four non-trivial
algorithmic methods (and corresponding software tools): The first is
\Untrendy, a system for flexibly but responsibly modeling the
combination of spacecraft photometric and stellar variability that
makes every \Kepler\ target a strong variable in the raw data stream.
The second is \turnstyle, a brute-force generative-modeling machinery
that creates and tests (against enormous numbers of \Kepler\ targets)
enormous numbers of exoplanetary hypotheses.  The third is \bart, an
inference package that returns responsible, properly marginalized
parameter estimates under parameterized (or non-parametric) prior
probability density functions for exoplanet system properties.  The
fourth is \TheCreator, a hierarchical modeling framework and software
system that permits probabilistic inference of the true distribution
of exoplanetary system parameters (``architecture'' in the lingo),
starting at parameter estimation and including all the information
coming not just from well measured systems but also from all the
(numerous) marginal and null detections among the \Kepler\ targets.

The common thread joining all of these projects is probabilistic
inference.  The co-Is are known in the astrophysics community for
creating useful tools and methods for principled inference, most
notably with generative, justified, causal models and Markov Chain
Monte Carlo (MCMC).  The \Untrendy\ system is novel because it is
based on a model that is very flexible where it needs to be, but
inflexible (relative to its competitors) where it doesn't.  The
\turnstyle\ machinery can make discoveries that predecessors have
missed because it makes principle approximations to full generative
modeling of the data without the enormous computational cost.  It also
capitalizes on current computer architecture where it can.  The
\bart\ package capitalizes on the co-Is expertise in fast, self-tuning
MCMC to perform parameter estimation efficiently.  Part of its core
design philosophy is flexibility in parameterization of systems; this
makes it easy to propagate its results---even from different systems
with very different multiplicities---into population inference.
\TheCreator\ makes use of concepts from probabilistic graphical models
(Bayesian networks) to incorporate ideas about causal mechanisms and
quantitatively precise noise models into global population inferences.

Each of the four packages embodies a set of important ideas in data
analysis and will be expressed in the form of easy-to-use, public,
open-source software.

\section{why re-analyze \Kepler\ data?}

...contribute to \Kepler\ efforts

...multiple eyes

...most important data set in the business

...this team has special and demonstrated skills

\section{de-trending}

...running median is too flexible most of the time, and not flexible enough when there are discontinuities

\section{fast, approximate hypothesis testing}

...fastness

\section{probabilistic parameter estimation}

...bart and emcee

\section{hierarchical population inference}

...pgm intro

\section{management plan}

...roles of co-Is

...schedule and deliverables

\end{document}
