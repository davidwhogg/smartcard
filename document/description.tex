% this file is part of the Smart Card project.

\documentclass[letterpaper,12pt,preprint]{hack_aastex}

\usepackage{amsmath}
\usepackage{color}
\usepackage[pagebackref=false]{hyperref}
\definecolor{linkcolor}{rgb}{0,0,0.25}
\hypersetup{colorlinks=true,linkcolor=linkcolor,citecolor=linkcolor,
            filecolor=linkcolor,urlcolor=linkcolor}
\newcommand{\hurl}[1]{{\scriptsize\url{#1}}}

\usepackage{epsfig}
\usepackage{graphicx}
\newcommand{\sectionname}{Section}
\setlength{\headheight}{2ex}
\setlength{\headsep}{2ex}
%----- exact 1-in margins
% NB: headheight and headsep MUST exist and be set
\setlength{\textwidth}{6.5in}
\setlength{\textheight}{9in}
\addtolength{\textheight}{-1.0\headheight}
\addtolength{\textheight}{-1.0\headsep}
\setlength{\topmargin}{0.0in}
\setlength{\oddsidemargin}{0.0in}
\setlength{\evensidemargin}{0.0in}

%----- typeset certain kinds of words
\newcommand{\observatory}[1]{\textsl{#1}}
\newcommand{\package}[1]{\textsf{#1}}
\newcommand{\project}[1]{\package{#1}}
\newcommand{\an}{\package{Astrometry.net}}
\newcommand{\NASA}{\observatory{NASA}}
\newcommand{\Kepler}{\observatory{Kepler}}
\newcommand{\kepler}{\Kepler}
\newcommand{\MAST}{\observatory{MAST}}
\newcommand{\EA}{\observatory{Exoplanet Archive}}
\newcommand{\TESS}{\observatory{TESS}}
\newcommand{\galex}{\observatory{GALEX}}
\newcommand{\Spitzer}{\observatory{Spitzer}}
\newcommand{\gaia}{\observatory{Gaia}}
\newcommand{\lsst}{\observatory{LSST}}
\newcommand{\sdss}{\observatory{SDSS}}
\newcommand{\latin}[1]{\textit{#1}}
\newcommand{\eg}{\latin{e.g.}}
\newcommand{\etal}{\latin{et~al.}}
\newcommand{\etc}{\latin{etc.}}
\newcommand{\ie}{\latin{i.e.}}
\newcommand{\vs}{\latin{vs.}}

%----- typeset journals
% \newcommand{\aj}{Astron.\,J.}
% \newcommand{\apj}{Astrophys.\,J.}
% \newcommand{\apjl}{Astrophys.\,J.\,Lett.}
% \newcommand{\apjs}{Astrophys.\,J.\,Supp.\,Ser.}
% \newcommand{\mnras}{Mon.\,Not.\,Roy.\,Ast.\,Soc.}
% \newcommand{\aap}{Astron.\,\&~Astrophys.}

%----- Tighten up paragraphs and lists
\setlength{\parskip}{0.0ex}
\setlength{\parindent}{0.2in}
\renewenvironment{itemize}{\begin{list}{$\bullet$}{%
  \setlength{\topsep}{0.0ex}%
  \setlength{\parsep}{0.0ex}%
  \setlength{\partopsep}{0.0ex}%
  \setlength{\itemsep}{0.0ex}%
  \setlength{\leftmargin}{1.5\parindent}}}{\end{list}}
\newcounter{actr}
\renewenvironment{enumerate}{\begin{list}{\arabic{actr}.}{%
  \usecounter{actr}%
  \setlength{\topsep}{0.0ex}%
  \setlength{\parsep}{0.0ex}%
  \setlength{\partopsep}{0.0ex}%
  \setlength{\itemsep}{0.0ex}%
  \setlength{\leftmargin}{1.5\parindent}}}{\end{list}}

%----- mess with paragraph spacing!
\makeatletter
\renewcommand\paragraph{\@startsection{paragraph}{4}{\z@}%
                                    {1ex}%
                                    {-1em}%
                                    {\normalfont\normalsize\bfseries}}
\makeatother

%----- Special Hogg list for references
  \newcommand{\hogglist}{%
    \rightmargin=0in
    \leftmargin=0.25in
    \topsep=0ex
    \partopsep=0pt
    \itemsep=0ex
    \parsep=0pt
    \itemindent=-1.0\leftmargin
    \listparindent=\leftmargin
    \settowidth{\labelsep}{~}
    \usecounter{enumi}
  }

%----- side-to-side figure macro
%------- make numbers add up to 94%
 \newlength{\figurewidth}
 \newlength{\captionwidth}
 \newcommand{\ssfigure}[3]{%
   \setlength{\figurewidth}{#2\textwidth}
   \setlength{\captionwidth}{\textwidth}
   \addtolength{\captionwidth}{-\figurewidth}
   \addtolength{\captionwidth}{-0.02\figurewidth}
   \begin{figure}[htb]%
   \begin{tabular}{cc}%
     \begin{minipage}[c]{\figurewidth}%
       \resizebox{\figurewidth}{!}{\includegraphics{#1}}%
     \end{minipage} &%
     \begin{minipage}[c]{\captionwidth}%
       \textsf{\caption[]{\footnotesize {#3}}}%
     \end{minipage}%
   \end{tabular}%
   \end{figure}}

%----- top-bottom figure macro
 \newlength{\figureheight}
 \setlength{\figureheight}{0.75\textheight}
 \newcommand{\tbfigure}[2]{%
   \begin{figure}[htp]%
   \resizebox{\textwidth}{!}{\includegraphics{#1}}%
   \textsf{\caption[]{\footnotesize {#2}}}%
   \end{figure}}

%----- deal with pdf page-size stupidity
\special{papersize=8.5in,11in}
\setlength{\pdfpageheight}{\paperheight}
\setlength{\pdfpagewidth}{\paperwidth}

% no more bad lines!
\sloppy

\newcommand{\kplr}{\package{kplr}}
\newcommand{\Untrendy}{\package{Untrendy}}
\newcommand{\Bart}{\package{Bart}}
\newcommand{\emcee}{\package{emcee}}
\newcommand{\TheCreator}{\package{TheCreator}}
\newcommand{\dd}{\,\mathrm{d}}
\newcommand{\bvec}[1]{{\ensuremath{{\boldsymbol{#1}}}}}
\pagestyle{myheadings}
\markright{\textsf{\footnotesize Hogg \& Foreman-Mackey / %
                   End-to-end probabilistic modeling of Kepler data}}

\usepackage{listings}
\lstset{%
    language=Python,
    basicstyle=\scriptsize\ttfamily,
    showspaces=false,
    showstringspaces=false,
    tabsize=2,
    breaklines=false,
    breakatwhitespace=true,
    identifierstyle=\ttfamily,
    keywordstyle=\bfseries\color[rgb]{0.133,0.545,0.133},
    commentstyle=\color[rgb]{0.4,0.4,0.4},
    stringstyle=\color[rgb]{0.627,0.126,0.941},
}

\begin{document}

\section{Introduction?}


\figurename~\ref{fig:planet-properties}.

\ssfigure{figures/planet_properties.pdf}{0.60}{%
The properties of confirmed and candidate exoplanets from \Kepler.
Data from NASA Exoplanet Archive.
\label{fig:planet-properties}}

\section{A data-driven model of the \Kepler\ pixels}

Despite the impressive precision, \Kepler's photometry is plagued by
substantial ``systematics'' due to instrumental effects (pointing shifts,
temperature variations, \etc) and real astrophysical signals (stellar
variability, transiting exoplanets, \etc).
It is of significant scientific interest to separate these two types of
signals and much progress has been made towards removing instrumental
systematics while robustly retaining the astrophysical effects (CITE PDC, ARC,
\etc).
These algorithms are all based on a fundamental \emph{causal} argument: the
signals that are common across nearby targets must be due to instrumental
variations because there can be no causal connection between the astrophysical
objects.
The idea is simple but a lot of the work that goes into implementing these
models involves combating over-fitting.

We propose to implement a method based on the same argument that models the
instrumental effects \emph{at the pixel level} instead of in the photometry.
This method makes a prediction for the variability caused by the instrument in
a specific pixel at a specific time by using the pixel time series of similar
nearby (but causally disconnected) targets at different times.
This prediction can then be used to remove or model the systematics in various
ways but our current thinking is that this model is best used TODO photometry
of prediction.

We'll model the flux due to systematic variations in pixel $k$ of target
$n$ at time $t$ as
\begin{eqnarray}
f_{nk}(t) &=& \bvec{c}^\mathrm{T}\cdot\bvec{f}_{\sim n}(t) + c_0
              + \epsilon_{nk}(t)
\end{eqnarray}
where $\bvec{f}_{\sim n}$ is the vector of some $K$ pixels around nearby
targets (not including $n$), $\bvec{c}$ is a vector of linear weights, $c_0$
is a constant DC level and $\epsilon_{nk}$ represents the stochastic pixel
noise.
There are many choices that must be made to evaluate this model but the main
ones are: (a) the number of pixels $K$ should be used in $\bvec{f}_{\sim n}$,
and (b) how should $\bvec{c}$ and $c_0$ are chosen.

\section{Optimized aperture photometry}

\section{Flexible, non-parametric modeling of light curve variations}

\section{The \Kepler\ focal plane}

\section{Management plan and schedule}

\clearpage
\begin{thebibliography}{}\raggedright%

\bibitem[Petigura \etal(2013)]{petigura}
Petigura, E.~A., Howard, A.~W., \& Marcy, G.~W.\ 2013,
Proceedings of the National Academy of Science, 110, 19273

\end{thebibliography}

\end{document}
