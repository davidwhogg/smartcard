% this file is part of the Smart Card project.

% to-do
% -----
% - outline
% - write
% - submit
% - spend

\documentclass[letterpaper,12pt]{article}
\usepackage{epsfig}
\usepackage{graphicx}
\newcommand{\sectionname}{Section}
\setlength{\headheight}{2ex}
\setlength{\headsep}{3ex}
%----- exact 1-in margins
% NB: headheight and headsep MUST exist and be set
\setlength{\textwidth}{6.5in}
\setlength{\textheight}{9in}
\addtolength{\textheight}{-1.0\headheight}
\addtolength{\textheight}{-1.0\headsep}
\setlength{\topmargin}{0.0in}
\setlength{\oddsidemargin}{0.0in}
\setlength{\evensidemargin}{0.0in}

%----- typeset certain kinds of words
\newcommand{\observatory}[1]{\textsl{#1}}
\newcommand{\package}[1]{\textsf{#1}}
\newcommand{\project}[1]{\package{#1}}
\newcommand{\an}{\package{Astrometry.net}}
\newcommand{\NASA}{\observatory{NASA}}
\newcommand{\Kepler}{\observatory{Kepler}}
\newcommand{\kepler}{\Kepler}
\newcommand{\MAST}{\observatory{MAST}}
\newcommand{\EA}{\observatory{Exoplanet Archive}}
\newcommand{\TESS}{\observatory{TESS}}
\newcommand{\galex}{\observatory{GALEX}}
\newcommand{\Spitzer}{\observatory{Spitzer}}
\newcommand{\gaia}{\observatory{Gaia}}
\newcommand{\lsst}{\observatory{LSST}}
\newcommand{\sdss}{\observatory{SDSS}}
\newcommand{\latin}[1]{\textit{#1}}
\newcommand{\eg}{\latin{e.g.}}
\newcommand{\etal}{\latin{et~al.}}
\newcommand{\etc}{\latin{etc.}}
\newcommand{\ie}{\latin{i.e.}}
\newcommand{\vs}{\latin{vs.}}

%----- typeset journals
% \newcommand{\aj}{Astron.\,J.}
% \newcommand{\apj}{Astrophys.\,J.}
% \newcommand{\apjl}{Astrophys.\,J.\,Lett.}
% \newcommand{\apjs}{Astrophys.\,J.\,Supp.\,Ser.}
% \newcommand{\mnras}{Mon.\,Not.\,Roy.\,Ast.\,Soc.}
% \newcommand{\aap}{Astron.\,\&~Astrophys.}

%----- Tighten up paragraphs and lists
\setlength{\parskip}{0.0ex}
\setlength{\parindent}{0.2in}
\renewenvironment{itemize}{\begin{list}{$\bullet$}{%
  \setlength{\topsep}{0.0ex}%
  \setlength{\parsep}{0.0ex}%
  \setlength{\partopsep}{0.0ex}%
  \setlength{\itemsep}{0.0ex}%
  \setlength{\leftmargin}{1.5\parindent}}}{\end{list}}
\newcounter{actr}
\renewenvironment{enumerate}{\begin{list}{\arabic{actr}.}{%
  \usecounter{actr}%
  \setlength{\topsep}{0.0ex}%
  \setlength{\parsep}{0.0ex}%
  \setlength{\partopsep}{0.0ex}%
  \setlength{\itemsep}{0.0ex}%
  \setlength{\leftmargin}{1.5\parindent}}}{\end{list}}

%----- mess with paragraph spacing!
\makeatletter
\renewcommand\paragraph{\@startsection{paragraph}{4}{\z@}%
                                    {1ex}%
                                    {-1em}%
                                    {\normalfont\normalsize\bfseries}}
\makeatother

%----- Special Hogg list for references
  \newcommand{\hogglist}{%
    \rightmargin=0in
    \leftmargin=0.25in
    \topsep=0ex
    \partopsep=0pt
    \itemsep=0ex
    \parsep=0pt
    \itemindent=-1.0\leftmargin
    \listparindent=\leftmargin
    \settowidth{\labelsep}{~}
    \usecounter{enumi}
  }

%----- side-to-side figure macro
%------- make numbers add up to 94%
 \newlength{\figurewidth}
 \newlength{\captionwidth}
 \newcommand{\ssfigure}[3]{%
   \setlength{\figurewidth}{#2\textwidth}
   \setlength{\captionwidth}{\textwidth}
   \addtolength{\captionwidth}{-\figurewidth}
   \addtolength{\captionwidth}{-0.02\figurewidth}
   \begin{figure}[htb]%
   \begin{tabular}{cc}%
     \begin{minipage}[c]{\figurewidth}%
       \resizebox{\figurewidth}{!}{\includegraphics{#1}}%
     \end{minipage} &%
     \begin{minipage}[c]{\captionwidth}%
       \textsf{\caption[]{\footnotesize {#3}}}%
     \end{minipage}%
   \end{tabular}%
   \end{figure}}

%----- top-bottom figure macro
 \newlength{\figureheight}
 \setlength{\figureheight}{0.75\textheight}
 \newcommand{\tbfigure}[2]{%
   \begin{figure}[htp]%
   \resizebox{\textwidth}{!}{\includegraphics{#1}}%
   \textsf{\caption[]{\footnotesize {#2}}}%
   \end{figure}}

%----- deal with pdf page-size stupidity
\special{papersize=8.5in,11in}
\setlength{\pdfpageheight}{\paperheight}
\setlength{\pdfpagewidth}{\paperwidth}

% no more bad lines!
\sloppy

\newcommand{\kplr}{\package{kplr.co}}
\newcommand{\Untrendy}{\package{Untrendy}}
\newcommand{\Turnstile}{\package{IronHorse}}
\newcommand{\Bart}{\package{Bart}}
\newcommand{\emcee}{\package{emcee}}
\newcommand{\TheCreator}{\package{TheCreator}}
\newcommand{\KIC}{\textsl{KIC}}
\newcommand{\KOI}{\textsl{KOI}}
\pagestyle{myheadings}
\markright{\textsf{\small Hogg \& Foreman-Mackey / probabilistic modeling of Kepler data}}
\begin{document}

\section{why re-analyze \Kepler\ data?}

The \Kepler\ team is one of the most successful collaborations in the
history of astronomy, and the \Kepler\ data are by no means trivial to
understand and use.  Why would even a plucky twosome consider
competing with this extremely capable group?  The answer is that this
is \emph{not} a proposal to \emph{compete} with the \Kepler\ team.
This is a proposal to enhance the capabilities of the \Kepler\ mission
and give new tools to the entire community to increase the scientific
return from the \Kepler\ data (and many related and future data sets).

The first reason that it makes sense to re-consider the \Kepler\ data
from the ground up is the simple point that it is valuable to have
multiple eyes on the problem.  We bring different prejudices,
different intuitions, and different expertise than exists on the
\Kepler\ team.  In addition, we can analyze the \Kepler\ data in a
more open-ended way since we do not bear the responsibility of
delivering results according to existing schedules and requirements.
That said, we very much hope to create methods and code that make the
\Kepler\ team products more valuable and easier to deliver.

The second reason it makes sense for us to take an independent look at
the \Kepler\ data is that we will bring a strong probabilistic,
causal, generative model approach to everything we do.  In each
component of this project, we are trying to write down a probability
for the data given the parameters, and in each component we are trying
to have the relationship of the parameters to the predictions obey
deep ideas we have about the physical or causal processes that
generate the data.  For certain kinds of (endearing) fanatics---or
perhaps more correctly, under certain kinds of restrictive
assumptions---these probabilistic approaches to data analysis are
guaranteed to succeed over more heuristic methods.  We have god on our
side, in this sense.  But because we are free from project
requirements, we can take a more principled approach to the data
analysis and see if that delivers better results.  If it does (because
we are proposing to generate public, open-source, easy-to-use code)
\emph{everybody wins.}

Finally, the most important reason to bring new people and new ideas
to the \Kepler\ data is that the \Kepler\ data are so damned
important.  The vast majority of known exoplanets are
\Kepler\ discoveries (CITE), and each \Kepler\ discovery brings with
it so much important information about each planetary system (CITE).
The data set has included many multiple-planet systems, and supported
preliminary studies of population statistics, such as multiplicity and
inclination distributions (CITE).  These studies have been made
possible in part by the scale and in part by the simplicity of the
\Kepler\ experiment; it has created a statistically useable sample.
Although \Kepler\ is just the first step of \NASA's exoplanet journey,
it is the best data set in it's class right now, and will retain
unique capabilities for many years to come.  Furthermore, many future
data sets (for example that from \TESS; CITE), produce data that are
\Kepler-like in many ways.  \textbf{If we can take the most valuable
  data set in exoplanetary astrophysics and make it substantially more
  valuable with an ADAP-scale project, we can deliver to
  \NASA\ outstanding science per dollar.}

Even if you are convinced that the data need to be re-analyzed, why
should they be reanalyzed by this team?  Although the proposers do not
have a long history in exoplanet science, we do have some experience,
breaking ground in the area of hierarchical modeling of exoplanet
populations (CITE Hogg), and telescope light-field modeling for
exoplanet direct detection and spectroscopy (CITE Oppenheimer).  Much
more importantly, the proposal team has enormous experience in two
areas that are crucial if we are going to squeeze more information out
of the \Kepler\ data stream: We have great experience with calibration
and sensitive statistical analysis of enormous astrophysical data
sests (CITE some eg papers), and we are among the world leaders in
probabilistic modeling of astrophysical data sets (CITE some eg
papers).  In particular, Co-I Hogg was partially responsible for the
precise self-calibration of the \observatory{Sloan Digital Sky Survey}
imaging data (CITE Padmanabhan), which in turn led (almost directly)
to discoveries such as ultra-faint Milky Way companions Willman I
(CITE) and XXX (CITE) and the detection of the baryon acoustic feature
in the large-scale structure (CITE).  These two
strengths---calibration and probabilistic modeling---are not unique to
the proposing team, but their combination is unusual and puts us in a
very strong position to deliver extremely valuable new methods to the
\Kepler\ team and larger community.

\section{an API for \Kepler\ data}

foo

\section{de-trending}

...running median is too flexible most of the time, and not flexible enough when there are discontinuities

...DFM: Explain what is currently done---or at least \emph{said to be done}.  With citations.

...DFM: Explain what \Untrendy\ will do in its first release.  And demo figure.

...Hogg:  Explain what \Untrendy\ might do in future releases.

\section{fast, approximate hypothesis testing}

...fastness

...DFM:  Explain what is currently done and how.  With citations.

...Hogg:  Explain what \Turnstile\ will do in its first release.  DFM:  And demo figure.

y...Hogg:  Explain what future releases of \Turnstile\ might do.

\section{probabilistic parameter estimation}

...DFM:  Explain the range of things that are currently done.  With citations and possibly also criticisms where appropriate.

...DFM:  Explain what \Bart\ will do in its first release.  And demo figures.

...Hogg:  What might \Bart\ do in future releases?

\section{hierarchical population inference}

...pgm intro

...DFM:  Explain what is currently done, perhaps concentrating on Tremaine paper and competitors.  Also we should touch on theory here.

...Hogg:  What will \TheCreator\ do?

\section{side projects}

...GALEX project

...false-positive detection and characterization

...variable-star science

\section{management plan}

This project is the PhD dissertation project of Co-I Foreman-Mackey;
in this sense Foreman-Mackey is the scientific lead of the project,
making decisions about scientific priorities and the ordering of
experiments and code enhancements.  PI Hogg is the faculty advisor on
this dissertation; in this sense Hogg is the management lead of the
project, ensuring that deadlines get met, papers get written, and code
gets properly documented and released.  The budget also includes a
small amount of support for an additional part-time graduate student
and postdoc.  These are budgeted to support short-term scentific
collaborations that make use of the software or improve it.  That is,
some of the sub-parts of the project can be executed by new graduate
students or postdocs at NYU.

The project is budgeted for three years of project lifetime.  An
approximate schedule is the following

\paragraph{first year:}
Version~1.0 release of \kplr\ and version~0.1 releases of \Untrendy,
\Turnstile, and \Bart, executing the first stages of approximation
identified above.  Submission of short papers describing these three
codes, comparing them with standard community practice, and showing
first results.  Execution of \Untrendy\ and \Turnstile\ on the entire
\KIC; publication of all candidate transiting exoplanets, especially
those previously unknown.  Execution of \Bart\ on the best or most
interesting of those companions, with an eye to exoplanet
characterization and false-positive rejection.

\paragraph{second year:}
Submission of papers describing the results of the \Untrendy,
\Turnstile, and \Bart\ from the first year.  Version~0.1 release of
\TheCreator, executing the simplest meaningful graphical model, as
described above.  Execution of \Bart\ and \TheCreator\ on the complete
set of exoplanet candidates geenerated in the first year.  Submission
of paper describing those results.  With feedback from this full
``closed loop'', re-prioritize enhancements to \Untrendy\ and
\Turnstile; add features to \Bart.  Version~0.2 code releases for
these packages.  Start side projects on stellar variability,
eclipsing binaries, transit-timing variations, or limb-darkening as
appropriate.

\paragraph{third year:}
Following up best hints in results from \TheCreator in the second
year, complexify or adjust the graphical model, or launch multiple
models for competition.  Re-run everything on the entire \KIC to
update the population analysis to incorporate all improvements.
Submit papers describing improved software and new results.  Complete
side projects and submit papers.  Make final decisions about final
code versions, final code adjustments, documentation, and hosting and
release Version~1.0 for all four of \Untrendy, \Turnstile, \Bart, and
\TheCreator, along with (possibly very short) papers on arXiv or ASCL.

\end{document}
