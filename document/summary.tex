% this file is part of the Smart Card project.

\documentclass[letterpaper,12pt]{article}
\usepackage{epsfig}
\usepackage{graphicx}
\newcommand{\sectionname}{Section}
%----- exact 1-in margins
% NB: headheight and headsep MUST exist and be set
\setlength{\textwidth}{6.5in}
\setlength{\textheight}{9in}
\addtolength{\textheight}{-1.0\headheight}
\addtolength{\textheight}{-1.0\headsep}
\setlength{\topmargin}{0.0in}
\setlength{\oddsidemargin}{0.0in}
\setlength{\evensidemargin}{0.0in}

%----- typeset certain kinds of words
\newcommand{\observatory}[1]{\textsl{#1}}
\newcommand{\package}[1]{\textsf{#1}}
\newcommand{\project}[1]{\package{#1}}
\newcommand{\an}{\package{Astrometry.net}}
\newcommand{\NASA}{\observatory{NASA}}
\newcommand{\Kepler}{\observatory{Kepler}}
\newcommand{\kepler}{\Kepler}
\newcommand{\MAST}{\observatory{MAST}}
\newcommand{\EA}{\observatory{Exoplanet Archive}}
\newcommand{\TESS}{\observatory{TESS}}
\newcommand{\galex}{\observatory{GALEX}}
\newcommand{\Spitzer}{\observatory{Spitzer}}
\newcommand{\gaia}{\observatory{Gaia}}
\newcommand{\lsst}{\observatory{LSST}}
\newcommand{\sdss}{\observatory{SDSS}}
\newcommand{\latin}[1]{\textit{#1}}
\newcommand{\eg}{\latin{e.g.}}
\newcommand{\etal}{\latin{et~al.}}
\newcommand{\etc}{\latin{etc.}}
\newcommand{\ie}{\latin{i.e.}}
\newcommand{\vs}{\latin{vs.}}

%----- typeset journals
% \newcommand{\aj}{Astron.\,J.}
% \newcommand{\apj}{Astrophys.\,J.}
% \newcommand{\apjl}{Astrophys.\,J.\,Lett.}
% \newcommand{\apjs}{Astrophys.\,J.\,Supp.\,Ser.}
% \newcommand{\mnras}{Mon.\,Not.\,Roy.\,Ast.\,Soc.}
% \newcommand{\aap}{Astron.\,\&~Astrophys.}

%----- Tighten up paragraphs and lists
\setlength{\parskip}{0.0ex}
\setlength{\parindent}{0.2in}
\renewenvironment{itemize}{\begin{list}{$\bullet$}{%
  \setlength{\topsep}{0.0ex}%
  \setlength{\parsep}{0.0ex}%
  \setlength{\partopsep}{0.0ex}%
  \setlength{\itemsep}{0.0ex}%
  \setlength{\leftmargin}{1.5\parindent}}}{\end{list}}
\newcounter{actr}
\renewenvironment{enumerate}{\begin{list}{\arabic{actr}.}{%
  \usecounter{actr}%
  \setlength{\topsep}{0.0ex}%
  \setlength{\parsep}{0.0ex}%
  \setlength{\partopsep}{0.0ex}%
  \setlength{\itemsep}{0.0ex}%
  \setlength{\leftmargin}{1.5\parindent}}}{\end{list}}

%----- mess with paragraph spacing!
\makeatletter
\renewcommand\paragraph{\@startsection{paragraph}{4}{\z@}%
                                    {1ex}%
                                    {-1em}%
                                    {\normalfont\normalsize\bfseries}}
\makeatother

%----- Special Hogg list for references
  \newcommand{\hogglist}{%
    \rightmargin=0in
    \leftmargin=0.25in
    \topsep=0ex
    \partopsep=0pt
    \itemsep=0ex
    \parsep=0pt
    \itemindent=-1.0\leftmargin
    \listparindent=\leftmargin
    \settowidth{\labelsep}{~}
    \usecounter{enumi}
  }

%----- side-to-side figure macro
%------- make numbers add up to 94%
 \newlength{\figurewidth}
 \newlength{\captionwidth}
 \newcommand{\ssfigure}[3]{%
   \setlength{\figurewidth}{#2\textwidth}
   \setlength{\captionwidth}{\textwidth}
   \addtolength{\captionwidth}{-\figurewidth}
   \addtolength{\captionwidth}{-0.02\figurewidth}
   \begin{figure}[htb]%
   \begin{tabular}{cc}%
     \begin{minipage}[c]{\figurewidth}%
       \resizebox{\figurewidth}{!}{\includegraphics{#1}}%
     \end{minipage} &%
     \begin{minipage}[c]{\captionwidth}%
       \textsf{\caption[]{\footnotesize {#3}}}%
     \end{minipage}%
   \end{tabular}%
   \end{figure}}

%----- top-bottom figure macro
 \newlength{\figureheight}
 \setlength{\figureheight}{0.75\textheight}
 \newcommand{\tbfigure}[2]{%
   \begin{figure}[htp]%
   \resizebox{\textwidth}{!}{\includegraphics{#1}}%
   \textsf{\caption[]{\footnotesize {#2}}}%
   \end{figure}}

%----- deal with pdf page-size stupidity
\special{papersize=8.5in,11in}
\setlength{\pdfpageheight}{\paperheight}
\setlength{\pdfpagewidth}{\paperwidth}

% no more bad lines!
\sloppy

\addtolength{\textheight}{\headheight}
\addtolength{\textheight}{\headsep}
\setlength{\headheight}{0ex}
\setlength{\headsep}{0ex}
\pagestyle{empty}
\begin{document}

The \Kepler\ Satellite data have been used to find thousands of exoplanets.
In particular, many small, rocky planets have been found.
We have yet, however, to find a true Earth Analog, or Earth-like planet on a
year-ish orbit around a Sun-like star.
This is not for lack of photons;
these discoveries will be exceedingly challenging
because of the small number of transits within the Mission lifetime,
the uncalibrated variations in spacecraft (s/c) temperature and pointing,
and the existence of ubiquitous, stochastic stellar variability.
Here we \textbf{propose to create four software systems}, each of which will
address one or more of these challenges.
We also \textbf{propose to run these software systems on the \Kepler\ Data,
providing back to the entire community value-added data} that are more valuable
for almost any scientific question.

The first---the \PLM---is a data-driven self-calibration of the
\Kepler\ Data pixel values.  
The idea is that (because the design of the Satellite is so simple)
the s/c variability lives in a low-dimensionality space of pointing,
temperature, focus, and maybe some integrals or derivatives of those.
Inasmuch as these variations imprint on the pixels, they should be shared among
collections of pixels.
The \PLM\ builds a predictive model for every one of the $10^6$-ish pixels at
every one of the $10^5$-ish read-out epochs, using a combination of a
train-and-test framework and strong regularization to avoid over-fitting.
This model can potentially model out essentially all of the s/c-induced
variability in \Kepler\ Data lightcurves.

The second---the \OWL---is a method for generating frequentist
photometric estimators that are optimized for signal-to-noise in the presence
of correlated pixel noise.
Independently read-out pixels \emph{effectively} show
covariant noise because the fluxes they see are deterministic functions
of s/c positional, focus, and temperature ``noise''.
This covariant noise can be measured empirically in the data; the \OWL\ finds a
(soft) aperture that is orthogonal to the principal ``eigenimages''
of correlated noise.

The third---\George---is a replacement for the standard \Kepler\ Data
exoplanet-fitting likelihood function (which is usually based on a weighted sum
of squared residuals or $\chi^2$).
\George\ replaces this likelihood with a covariant noise model that accounts for
stochastic stellar variability.
Under the hood, it contains a flexible Gaussian Process, which effectively
builds a non-parametric variable lightcurve model and analytically marginalizes
it out.
The new parameters (beyond standard exoplanet transit parameters) introduced
by \George\ are only hyper-parameters setting the variance and correlation-time
properties of the stellar variability; these can be set rigidly or else inferred
simultaneously with all the exoplanet parameters.
Importantly, \George\ improves exoplanet inferences but is also fast enough that
it can be included in transiting-exoplanet search systems.

The fourth---\kpsf---is a full causal, generative model for the entire \Kepler\
focal plane.
This includes a parameterized model of the PSF, sub-pixel flat-field,
and s/c pointing.
The \kpsf\ provides a likelihood function for everything, so it
creates photometric estimators that saturate the Cram\'er--Rao bounds,
and also can
be inserted into full probabilistic (Bayesian) inference.
The output of \kpsf, which would run on the data calibrated by some limited
version of the \PLM, will in turn inform the \OWL\ and create lightcurves for
\George.
That is, full sensitivity will be achieved with all four components working
together.

Each of the four packages is not just code for \Kepler\ but also embodies
important ideas in data analysis.
Each will be released as easy-to-use, easy-to-modify, public, open-source
software, released under the MIT open-source license.
The idea is to make all results of the project easily useable, reproducible, and
extendable by anyone.

\end{document}
