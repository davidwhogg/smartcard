% this file is part of the Smart Card project.

\documentclass[letterpaper,12pt]{article}
\usepackage{epsfig}
\usepackage{graphicx}
\newcommand{\sectionname}{Section}
%----- exact 1-in margins
% NB: headheight and headsep MUST exist and be set
\setlength{\textwidth}{6.5in}
\setlength{\textheight}{9in}
\addtolength{\textheight}{-1.0\headheight}
\addtolength{\textheight}{-1.0\headsep}
\setlength{\topmargin}{0.0in}
\setlength{\oddsidemargin}{0.0in}
\setlength{\evensidemargin}{0.0in}

%----- typeset certain kinds of words
\newcommand{\observatory}[1]{\textsl{#1}}
\newcommand{\package}[1]{\textsf{#1}}
\newcommand{\project}[1]{\package{#1}}
\newcommand{\an}{\package{Astrometry.net}}
\newcommand{\NASA}{\observatory{NASA}}
\newcommand{\Kepler}{\observatory{Kepler}}
\newcommand{\kepler}{\Kepler}
\newcommand{\MAST}{\observatory{MAST}}
\newcommand{\EA}{\observatory{Exoplanet Archive}}
\newcommand{\TESS}{\observatory{TESS}}
\newcommand{\galex}{\observatory{GALEX}}
\newcommand{\Spitzer}{\observatory{Spitzer}}
\newcommand{\gaia}{\observatory{Gaia}}
\newcommand{\lsst}{\observatory{LSST}}
\newcommand{\sdss}{\observatory{SDSS}}
\newcommand{\latin}[1]{\textit{#1}}
\newcommand{\eg}{\latin{e.g.}}
\newcommand{\etal}{\latin{et~al.}}
\newcommand{\etc}{\latin{etc.}}
\newcommand{\ie}{\latin{i.e.}}
\newcommand{\vs}{\latin{vs.}}

%----- typeset journals
% \newcommand{\aj}{Astron.\,J.}
% \newcommand{\apj}{Astrophys.\,J.}
% \newcommand{\apjl}{Astrophys.\,J.\,Lett.}
% \newcommand{\apjs}{Astrophys.\,J.\,Supp.\,Ser.}
% \newcommand{\mnras}{Mon.\,Not.\,Roy.\,Ast.\,Soc.}
% \newcommand{\aap}{Astron.\,\&~Astrophys.}

%----- Tighten up paragraphs and lists
\setlength{\parskip}{0.0ex}
\setlength{\parindent}{0.2in}
\renewenvironment{itemize}{\begin{list}{$\bullet$}{%
  \setlength{\topsep}{0.0ex}%
  \setlength{\parsep}{0.0ex}%
  \setlength{\partopsep}{0.0ex}%
  \setlength{\itemsep}{0.0ex}%
  \setlength{\leftmargin}{1.5\parindent}}}{\end{list}}
\newcounter{actr}
\renewenvironment{enumerate}{\begin{list}{\arabic{actr}.}{%
  \usecounter{actr}%
  \setlength{\topsep}{0.0ex}%
  \setlength{\parsep}{0.0ex}%
  \setlength{\partopsep}{0.0ex}%
  \setlength{\itemsep}{0.0ex}%
  \setlength{\leftmargin}{1.5\parindent}}}{\end{list}}

%----- mess with paragraph spacing!
\makeatletter
\renewcommand\paragraph{\@startsection{paragraph}{4}{\z@}%
                                    {1ex}%
                                    {-1em}%
                                    {\normalfont\normalsize\bfseries}}
\makeatother

%----- Special Hogg list for references
  \newcommand{\hogglist}{%
    \rightmargin=0in
    \leftmargin=0.25in
    \topsep=0ex
    \partopsep=0pt
    \itemsep=0ex
    \parsep=0pt
    \itemindent=-1.0\leftmargin
    \listparindent=\leftmargin
    \settowidth{\labelsep}{~}
    \usecounter{enumi}
  }

%----- side-to-side figure macro
%------- make numbers add up to 94%
 \newlength{\figurewidth}
 \newlength{\captionwidth}
 \newcommand{\ssfigure}[3]{%
   \setlength{\figurewidth}{#2\textwidth}
   \setlength{\captionwidth}{\textwidth}
   \addtolength{\captionwidth}{-\figurewidth}
   \addtolength{\captionwidth}{-0.02\figurewidth}
   \begin{figure}[htb]%
   \begin{tabular}{cc}%
     \begin{minipage}[c]{\figurewidth}%
       \resizebox{\figurewidth}{!}{\includegraphics{#1}}%
     \end{minipage} &%
     \begin{minipage}[c]{\captionwidth}%
       \textsf{\caption[]{\footnotesize {#3}}}%
     \end{minipage}%
   \end{tabular}%
   \end{figure}}

%----- top-bottom figure macro
 \newlength{\figureheight}
 \setlength{\figureheight}{0.75\textheight}
 \newcommand{\tbfigure}[2]{%
   \begin{figure}[htp]%
   \resizebox{\textwidth}{!}{\includegraphics{#1}}%
   \textsf{\caption[]{\footnotesize {#2}}}%
   \end{figure}}

%----- deal with pdf page-size stupidity
\special{papersize=8.5in,11in}
\setlength{\pdfpageheight}{\paperheight}
\setlength{\pdfpagewidth}{\paperwidth}

% no more bad lines!
\sloppy

\addtolength{\textheight}{\headheight}
\addtolength{\textheight}{\headsep}
\setlength{\headheight}{0ex}
\setlength{\headsep}{0ex}
\newcommand{\kplr}{\package{kplr}}
\newcommand{\Untrendy}{\package{Untrendy}}
\newcommand{\Bart}{\package{Bart}}
\newcommand{\emcee}{\package{emcee}}
\newcommand{\TheCreator}{\package{TheCreator}}
\newcommand{\KIC}{\textsl{KIC}}
\newcommand{\KOI}{\textsl{KOI}}
\pagestyle{empty}
\begin{document}

The \Kepler\ observatory has provided the most important and
scientifically productive data set ever for the study of exoplanets;
it has yielded enormous numbers of new exoplanets and exoplanetary
systems; it has generated unprecedented discoveries, measurements, and
insights in the domain of exoplanet science.  This proposal is to
systematically re-analyze all of the \Kepler\ data from the ground up,
to verify and improve upon the tremendously successful
\Kepler-team-led analyses of the data, and to provide new inference
capabilities to the entire astrophysics community.  In particular,
\textbf{the proposal is to design, build, and release to the public
  four modular, novel, easy-to-use tools for probabilistic inference
  of exoplanet populations in the \Kepler\ data.}

The project team will design, build, and release four non-trivial
algorithmic methods (and corresponding software tools): The first is
\kplr, an application programming interface to the public
\Kepler\ data---and related and corresponding data from other
investigators, surveys, and missions---available at the Mikulski
Archive at Space Telescope and the Exoplanet Archive, making the data
and \Kepler\ pipeline outputs easier to fetch and manipulate with
automated software systems.  The second is \Untrendy, a system for
flexibly but responsibly modeling the combination of tiny spacecraft
photometric issues and stellar variability that makes every
\Kepler\ target variable even in the calibrated data stream.  These
effects, if unmodeled, are far larger than the exoplanet transits of
greatest importance.  The third is \Bart, an inference package that
returns responsible, properly marginalized parameter estimates under
parameterized (or non-parametric) prior probability density functions
for exoplanet system properties.  The fourth is \TheCreator, a
hierarchical modeling framework and software system that permits
probabilistic inference of the true distribution of exoplanetary
system parameters (``architecture'' in the lingo), starting at
parameter estimation and including all the information coming not just
from well measured systems but also from all the (numerous) marginal
and null detections among the \Kepler\ targets.

The common thread joining all of these projects is probabilistic
inference.  The project team members are known for creating useful
tools and methods for principled inference, most notably with
generative, justified, causal models and Markov Chain Monte Carlo
(MCMC).  The \Untrendy\ system is novel because it is based on a model
that is very flexible where it needs to be, but inflexible (relative
to its competitors) where it doesn't.  The \Bart\ package capitalizes
on the project team's expertise in fast, self-tuning MCMC to perform
parameter estimation efficiently.  Part of its core design philosophy
is flexibility in parameterization of systems; this makes it easy to
propagate results---even from different systems with very different
multiplicities---into population inference.  Also, the flexibility of
the model, particularly as regards limb-darkening profiles for the
host stars, makes it good for false-positive detection.
\TheCreator\ makes use of concepts from probabilistic graphical models
(Bayesian networks) to incorporate ideas about causal mechanisms and
quantitatively precise noise models into global population inferences.
Because it builds on the other components of the project, it has
flexibility to use the heterogeneous data obtainable through \kplr,
and work in different ``projections'' of the parameter space available
through \Bart.

Each of the four packages embodies a set of important ideas in data
analysis and will be expressed in the form of easy-to-use,
easy-to-modify, public, open-source software, released under an
open-source license.  The idea is to make all results of the project
easily reproducible and extendable by anyone.

\end{document}
